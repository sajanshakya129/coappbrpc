This library is developed to create a R\+PC that utilises features of Co\+AP protocol which uses U\+DP for transport. Protobuf is used as Interface Definition language(\+I\+D\+L). The goal is to use this R\+PC library over Constrained IoT devices.

\subsection*{Getting Started}

The following steps are written with the consideration that the system for development is fresh newly installed Linux Based System.

\subsubsection*{Prerequisites}

Please install following applications to build and install coappbrpc. 
\begin{DoxyCode}
sudo apt-get install cmake build-essential dh-autoreconf python python-pip
\end{DoxyCode}
 \paragraph*{Protocol buffer library Installation}

Follow the following link and https\+://github.com/protocolbuffers/protobuf/blob/master/src/\+R\+E\+A\+D\+M\+E.\+md \char`\"{}install protobuf\char`\"{} Or follow the following steps 
\begin{DoxyCode}
git clone https://github.com/protocolbuffers/protobuf.git
cd protobuf
git submodule update --init --recursive
./autogen.sh
./configure
make
make check
sudo make install
sudo ldconfig # refresh shared library cache.
\end{DoxyCode}


\#\#\#\# Lib\+Co\+AP Library Installation 
\begin{DoxyCode}
git clone https://github.com/obgm/libcoap.git
cd libcoap
\end{DoxyCode}
 O\+P\+T\+I\+O\+N\+AL\+:You might need to install pkg-\/config in your system if \textquotesingle{}pkg-\/config\textquotesingle{} is not found. 
\begin{DoxyCode}
sudo apt-get install pkg-config
\end{DoxyCode}
 Then configure libcoap and install libcoap 
\begin{DoxyCode}
./autogen.sh
./configure --disable-doxygen --disable-manpages
make
sudo make install
sudo ldconfig /usr/local/lib
\end{DoxyCode}


\#\# Installation of Co\+A\+P\+Pbrpc library 
\begin{DoxyCode}
git clone --single-branch -b newbranch https://github.com/sajanshakya129/coappbrpc
mkdir build && cd build
cmake ..
make
sudo make install
\end{DoxyCode}


\subsection*{Running Example}

Go to example/src folder and generate stub files required using following commands. 
\begin{DoxyCode}
cd [Project Folder]/example/
coappbrpc.sh rpc\_ping.proto
\end{DoxyCode}
 Client\+Stub.\+sh \hyperlink{ClientStub_8h_source}{Client\+Stub.\+h} are autogenerated.

\#\#\# Compiling and Running using cmake 
\begin{DoxyCode}
mkdir build && cd build && cmake ..
make
\end{DoxyCode}
 \paragraph*{Running client}

In example folder, bin folder is created with executable files i.\+e. client and server 
\begin{DoxyCode}
./bin/client
\end{DoxyCode}
 \#\#\#\# Running Server 
\begin{DoxyCode}
./bin/server
\end{DoxyCode}
 OR

\#\#\# Compiling and Running Client from terminal manually 
\begin{DoxyCode}
g++ -o client rpc\_ping.pb.cc ClientStub.cc client.cpp -lcoappbrpc -lcoap-2 -lprotobuf -lpthread
./client
\end{DoxyCode}
 \#\#\# Compiling and Running Server from terminal manually 
\begin{DoxyCode}
g++ -o server rpc\_ping.pb.cc server.cpp -lcoappbrpc -lcoap-2 -lprotobuf -lpthread
./server
\end{DoxyCode}


\subsection*{Creating Proto File}

To create a service for R\+PC, you need to define the response and request schema along with defination of Service in protofile as shown below. 
\begin{DoxyCode}
\{c++\}
syntax = "proto3"; // Denotes that we are using Protocol buffers 3 for compiling

package coappbrpc.api; // Packaging codes under coappbrpc.api namespace

option cc\_generic\_services = true;// To use protocol buffers services

message PingRequest \{  // User defined Request named "PingRequest" with parameter "msg". 
    string msg = 1; // For multiple parameters the numbering is done in increasing format 
    //which is non-repititive or cannot be duplicate.
\}

message PingResponse \{  // User defined Response named "PingResponse" with parameter "result". 
    string result = 1;
\}

service PingService \{ //User defined Service named "PingService" 
    rpc Ping (PingRequest) returns (PingResponse); //User defined method named "Ping" which takes
       "PingRequest" as input 
    //and give "PingResponse" as output.
\}
\end{DoxyCode}


\#\# Creating Client 
\begin{DoxyCode}
\{c++\}
#include <coappbrpc/ClientRPC.h> // To Create client you must include this file: coappbrpc/ClientRPC.h

#include "ClientStub.h" //including ClientStub.h generated when you run "coappbrpc.sh <protofile>" in
       command prompt.
#include "rpc\_ping.pb.h" //including rpc\_ping.pb.h generate when you run "coappbrpc.sh <protofile>" 
//in command prompt. <filename.proto> will generate "filename.pb.h" and "filename.pb.cc"

#include <string>

using ::coappbrpc::ClientRPC;  //Must be included in code
using ::coappbrpc::api::PingRequest; //Must be included in code
using ::coappbrpc::api::PingResponse; //Must be included in code
using ::coappbrpc::api::PingService; //Must be included in code
using namespace std;

class PingClient \{ //Defining Client Class
public:
  string ping(const string &msg) \{ //Defining method ping where we use RPC call

    PingRequest request;
    request.set\_msg(msg); //Setting user msg to request. 
    //Note: In set\_msg, msg is the parameter that we defined in protofile request.

    PingResponse response;
    stub.Ping(request, &response); // This is where actual RPC call is done.
   //Note: Ping is the method that we defined in protofile.

    return response.result();
  \}

private:
    ClientStub stub; //creating instance of Client Stub
\};

int main(void) \{
  GOOGLE\_PROTOBUF\_VERIFY\_VERSION; //verifiying protocol buffers version
  ClientRPC *client = ClientRPC::getInstance(); //creating a client instance
  client->setServerAddr("localhost:5683"); //setting up Server address. Default is "localhost:5683". 
                                        //Note: must be in format <server\_address>:<port\_no>
  std::string msg("Hello world"); //defining msg to be sent to Server.

  PingClient pclient;
  std::string reply = pclient.ping(msg);
  std::cout << "Ping received: " << reply << std::endl;
\}
\end{DoxyCode}


\#\# Creating Server 
\begin{DoxyCode}
\{c++\}
#include <coappbrpc/ServerRPC.h> //To create server you must include this file: coappbrpc/ServerRPC.h
#include "rpc\_ping.pb.h" //including rpc\_ping.pb.h generate when you run "coappbrpc.sh <protofile>" in
       command prompt.
                        //<filename.proto> will generate "filename.pb.h" and "filename.pb.cc"

using ::coappbrpc::ServerRPC; //Must be included in code

namespace coappbrpc \{
namespace api \{

using ::google::protobuf::Closure; //Must be included in code
using ::google::protobuf::RpcController; //Must be included in code

class PingServiceImpl : public PingService \{ //User defined class name PingServiceImpl which inherits from
       PingService class.
public:
  PingServiceImpl()\{\};

  virtual void Ping(RpcController *controller, const PingRequest *request,
                    PingResponse *response, Closure *done) \{ 
    //Defining Function named "Ping" which is defined as method in protobuf file. This is where you do 
    //processing and generate result and set result. 

    // Do your processing here

    //Setting  result and accessing request parameter "msg".

    response->set\_result("I got your message: " + request->msg()); 
  \}
\};

\} // namespace api
\} // namespace coappbrpc

int main() \{
  ServerRPC server; //Creating instance of server
  server.registerService(new ::coappbrpc::api::PingServiceImpl()); //Registering service
  server.runServer(); //running server
  // server.runServer("localhost:5683");// You can run server by providing server\_address and port no.
  return 0;
\}
\end{DoxyCode}


\subsection*{Uninstall library}

To uninstall library and files related from your system. you need to execute following command. 
\begin{DoxyCode}
sudo uninstall\_coappbrpc.sh
\end{DoxyCode}


\subsection*{Authors}


\begin{DoxyItemize}
\item {\bfseries Sajan Shakya} -\/ {\itshape Initial work} -\/ \href{https://github.com/sajanshakya129}{\tt Github}
\end{DoxyItemize}

\subsection*{References}

This library was created referring to library \href{https://github.com/madwyn/libpbrpc}{\tt Lib\+Pbrpc} and g\+R\+PC.

\subsection*{License}

This project is licensed under the M\+IT License -\/ see the \hyperlink{md_LICENSE}{L\+I\+C\+E\+N\+SE.md} file for details 